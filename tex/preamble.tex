\usepackage{babel}

\usepackage{titling}

\pretitle{\begin{center}\huge\sffamily\bfseries}
\posttitle{\end{center}}

\preauthor{\begin{center}\Large\sffamily\begin{tabular}[t]{c}}
\postauthor{\end{tabular}\end{center}}

\predate{\begin{center}\Large\sffamily}
\postdate{\end{center}}

\usepackage[margin=2.54cm]{geometry}
\setlength{\marginparwidth}{2cm} % Para el paquete 'todonotes'

\usepackage[svgnames]{xcolor}

\usepackage{titlesec}

\titleformat{\part}[display]{\Huge\bfseries\filcenter}
  {\color{MidnightBlue}\scalebox{2.5}{\thepart}}
  {1ex}{}

\titleformat{\chapter}{\huge\sffamily\bfseries}
  {\color{MidnightBlue}\scalebox{2.5}{\thechapter}}
  {1ex}{}

\titleformat{\section}{\LARGE\sffamily\bfseries}
  {\color{MidnightBlue}{\normalfont\sffamily\S}\thesection}
  {1ex}{}

\titleformat{\subsection}{\Large\sffamily\bfseries}
  {\color{MidnightBlue}{\normalfont\sffamily\S}\thesubsection}
  {1ex}{}

\usepackage{hyperref}
\hypersetup{
  colorlinks=true,
}

\usepackage{graphicx}

\usepackage{enumitem}
\newlist{dudas}{itemize}{2}
\setlist[dudas]{label=$\square$}
\newcommand{\cmark}{\rlap{$\checkmark$}\square}

\usepackage{todonotes}
\makeatletter
\renewcommand{\listoftodos}[1][\@todonotes@todolistname]{%
  \@ifundefined{chapter}{\section{#1}}{\section{#1}} \@starttoc{tdo}
}
\newcommand{\duda}[1]{%
  \todo[color=ForestGreen!50!]{#1}
}
\makeatother

\usepackage{mathtools}

\usepackage{amssymb}

\usepackage{amsthm}
\usepackage{thmtools}

\declaretheoremstyle[
  headfont=\sffamily\bfseries\color{MediumBlue},
  headpunct={},
  postheadspace=1em
]{thmblue}
\declaretheoremstyle[
  headfont=\sffamily\bfseries\color{FireBrick},
  headpunct={:}
]{thmred}
\declaretheoremstyle[
  headfont=\sffamily\bfseries\color{ForestGreen},
  headpunct={ ---}
]{thmgreen}
\declaretheoremstyle[
  headfont=\sffamily\bfseries\color{black},
]{thmblack}

\declaretheorem[name=Teorema, style=thmblue, numberwithin=section]{theorem}
\declaretheorem[name=Teorema, style=thmblue, numbered=no]{theorem*}
\declaretheorem[name=Proposición, style=thmblue, sibling=theorem]{proposition}
\declaretheorem[name=Proposición, style=thmblue, numbered=no]{proposition*}
\declaretheorem[name=Corolario, style=thmblue, sibling=theorem]{corollary}
\declaretheorem[name=Corolario, style=thmblue, numbered=no]{corollary*}
\declaretheorem[name=Lema, style=thmblue, sibling=theorem]{lemma}
\declaretheorem[name=Lema, style=thmblue, numbered=no]{lemma*}
\declaretheorem[name=Conjetura, style=thmblue, sibling=theorem]{conjecture}
\declaretheorem[name=Conjetura, style=thmblue, numbered=no]{conjecture*}
\declaretheorem[name=Definición, style=thmred, sibling=theorem]{definition}
\declaretheorem[name=Definición, style=thmred, numbered=no]{definition*}
\declaretheorem[name=Notación, style=thmred, sibling=theorem]{notation}
\declaretheorem[name=Notación, style=thmred, numbered=no]{notation*}
\declaretheorem[name=Ejemplo, style=thmgreen, sibling=theorem]{example}
\declaretheorem[name=Ejemplo, style=thmgreen, numbered=no]{example*}
\declaretheorem[name=Nota, style=thmgreen, sibling=theorem]{remark}
\declaretheorem[name=Nota, style=thmgreen, numbered=no]{remark*}
\declaretheorem[name=Problema, style=thmblack,]{problem}
\declaretheorem[name=Problema, style=thmblack, numbered=no]{problem*}
\declaretheorem[name=Custión, style=thmblack, sibling=problem]{question}
\declaretheorem[name=Custión, style=thmblack, numbered=no]{question*}
\declaretheorem[name=Ejercicio, style=thmblack, sibling=problem]{exercise}
\declaretheorem[name=Ejercicio, style=thmblack, numbered=no]{exercise*}

\newenvironment{solution}{%
  \begin{proof}[Solución] \color{MidnightBlue!80!black} \renewcommand{\qed}{}%
}{%
  \end{proof}%
}
