\chapter{Dudas}

\listoftodos[Dudas en el texto]

\section{Otras dudas}

\begin{dudas}
  \item Tras la definición de $\sigma$-álgebra de conjuntos se dice que la familia $\partsof{\Omega}$ es la $\sigma$-álgebra de conjuntos más grande posible sobre $\Omega$. No obstante, si tomamos un subconjunto cualquiera $\Omega^\prime \subset \Omega$, tenemos que $\partsof{\Omega}$ define una $\sigma$-álgebra de conjuntos sobre $\Omega^\prime$, que es, en efecto, más grande que $\partsof{\Omega^\prime}$, lo que contradice la afirmación anterior. ¿En qué sentido es $\partsof{\Omega}$ la $\sigma$-álgebra de conjuntos más grande posible sobre $\Omega$? ¿Para que una $\sigma$-álgebra de conjuntos proporcionara realmente una estructura adecuada para definir un espacio de probabilidades no debería aclarar que el conjunto más grande (en un sentido de inclusión) fuera el propio $\Omega$?
  \item No acabo de entender la \textit{filosofía} de la probabilidad condicionada y qué quiere decir en particular el teorema de Bayes. Por ello, me he visto el vídeo \guillemot{\textit{Bayes theorem, the geometry of changing beliefs}} de \textit{3blue1brown} (\url{https://www.youtube.com/watch?v=HZGCoVF3YvM}). Creo que ahora tengo un poco más de intuición sobre el tema. No obstante, me gustaría entender mejor el teorema y el concepto de probabilidad condicionada. Además, ¿por qué la fórmula del vídeo y de los apuntes tiene otra forma?
\end{dudas}
